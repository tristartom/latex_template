\usepackage{graphicx, multirow}

%customized packages
\usepackage{tabularx, multirow, rotating} %tables, dependent on multirow package.
%rotating is used for \begin{sideways} environment
\usepackage{subfig} %figures
\usepackage{algorithm} %algorithm
\usepackage{algpseudocode} %algorithm: use algorithmicx package, to refer to http://en.wikibooks.org/wiki/LaTeX/Algorithms_and_Pseudocode#Typesetting_using_the_algorithmicx_package
%TOREMOVE \usepackage[noend]{algorithmic} %algorithm
\usepackage{ulem} % strike out text
\usepackage{color} % colored text

%for listing, like SQL
\usepackage{listings} 
\lstdefinestyle{Oracle}{basicstyle=\ttfamily,
                        keywordstyle=\lstuppercase,
                        emphstyle=\itshape,
                        showstringspaces=true,
                        }
\makeatletter
\newcommand{\lstuppercase}{\uppercase\expandafter{\expandafter\lst@token
                           \expandafter{\the\lst@token}}}
\newcommand{\lstlowercase}{\lowercase\expandafter{\expandafter\lst@token
                           \expandafter{\the\lst@token}}}
\makeatother


%for circled text
%f: http://tex.stackexchange.com/questions/7032/good-way-to-make-textcircled-numbers
\usepackage{tikz}
\newcommand*\circled[1]{\tikz[baseline=(char.base)]{
            \node[shape=circle,draw,inner sep=1pt] (char) {#1};}} 

%A white number inside a black circle.
% \usepackage{tikz}
\newcommand{\ballnumber}[1]{\tikz[baseline=(myanchor.base)] \node[circle,fill=.,inner sep=1pt] (myanchor) {\color{-.}\bfseries\footnotesize #1};}
%% use it
%\ballnumber{37}

%for boldening algorithm line numbers.
%f: http://tex.stackexchange.com/questions/99493/how-to-make-the-line-number-in-algorithm-shown-in-bold/99515#99515
\newif\ifboldnumber
\newcommand{\boldnext}{\global\boldnumbertrue}
% Default definition is \footnotesize#1:
\algrenewcommand\alglinenumber[1]{%
  \footnotesize\ifboldnumber\bfseries\fi\global\boldnumberfalse#1:}

%for rcases, or right brace
\newenvironment{rcases}
  {\left.\begin{aligned}}
  {\end{aligned}\right\rbrace}

%color table
\usepackage{color, colortbl}
%\definecolor{name}{system}{definition}
\definecolor{Gray}{gray}{0.9}
\definecolor{LightCyan}{rgb}{0.88,1,1}
\usepackage[first=0,last=9]{lcg}
\newcommand{\ra}{\rand0.\arabic{rand}}

%diagnal cross in table
%also require tikz package
\usepackage{colortbl}
\usetikzlibrary{calc}
\usepackage{zref-savepos}

\newcounter{NoTableEntry}
\renewcommand*{\theNoTableEntry}{NTE-\the\value{NoTableEntry}}

\newcommand*{\notableentry}{%
  \multicolumn{1}{@{}c@{}|}{%
    \stepcounter{NoTableEntry}%
    \vadjust pre{\zsavepos{\theNoTableEntry t}}% top
    \vadjust{\zsavepos{\theNoTableEntry b}}% bottom
    \zsavepos{\theNoTableEntry l}% left
    \hspace{0pt plus 1filll}%
    \zsavepos{\theNoTableEntry r}% right
    \tikz[overlay]{%
      \draw[red]
        let
          \n{llx}={\zposx{\theNoTableEntry l}sp-\zposx{\theNoTableEntry r}sp},
          \n{urx}={0},
          \n{lly}={\zposy{\theNoTableEntry b}sp-\zposy{\theNoTableEntry r}sp},
          \n{ury}={\zposy{\theNoTableEntry t}sp-\zposy{\theNoTableEntry r}sp}
        in
        (\n{llx}, \n{lly}) -- (\n{urx}, \n{ury})
        (\n{llx}, \n{ury}) -- (\n{urx}, \n{lly})
      ;
    }% 
  }%
}

